% Compile in XeLaTeX
\documentclass[usenames,dvipsnames,11pt,aspectratio=169,t]{beamer} 

% Preamble {{{ 
% Note that [usenames,dvipsnames] is MANDATORY due to compatibility
% issues between tikz and xcolor packages.
\usepackage{verbatim}
\usetheme{ALICE}
\usepackage{polyglossia}
    \setmainlanguage{english}
\usepackage{csquotes}
\usepackage[font=tiny]{caption}
\usepackage{amsmath}
\usepackage{siunitx}
\usepackage{tikz}
    \usetikzlibrary{patterns}
    \usetikzlibrary{decorations.pathreplacing}
    \usetikzlibrary{arrows.meta,backgrounds,fit,positioning,petri,automata}
    \tikzstyle{block} = [rectangle, draw, fill=ABlue,
    text width=10em, text centered, rounded corners, minimum height=4em]
    \usetikzlibrary{shapes}
    \usetikzlibrary{backgrounds} % So ALICE logo is in front of line
    \newcommand\ngram[3][]{%
\path[#1] (0  :#3) -- ( 36:#2)
       -- (72 :#3) -- (108:#2)
       -- (144:#3) -- (180:#2)
       -- (216:#3) -- (252:#2)
       -- (288:#3) -- (324:#2)--cycle;}
\usepackage{subcaption}
\usepackage{booktabs,tabularx,colortbl}
\usepackage{enumitem}
\newlist{enum}{enumerate}{3}  
\setlist[enum, 1]{label*=\color{ARed}\arabic*., listparindent=21pt, font=\footnotesize, before*=\normalsize } 
\setlist[enum, 2]{label*=\color{ARed}\arabic*., listparindent=21pt, font=\tiny, before*=\footnotesize }
\setlist[itemize,1]{label=\color{ARed}\raisebox{0.25ex}{\tiny\textbullet}, before*=\usebeamerfont{light}\normalsize}
\setlist[itemize,2]{label=\color{ARed}\raisebox{0.25ex}{\tiny\textbullet}, before*=\usebeamerfont{light}\footnotesize}

\usepackage{dcolumn}
\usepackage{float}
\usepackage{lipsum}
\usepackage{listings}
\lstset{% general command to set parameter(s)
    basicstyle=\color{ARed}\footnotesize\ttfamily,
    keywordstyle=\color{black}\bfseries,
    identifierstyle=,
    commentstyle=\color{gray},
    stringstyle=\ttfamily,
    showstringspaces=false}

%\usepackage[ {{{
%    backend=biber,	% use biber backend (bibtex replacement) or bibtex
%    style=numeric,	% Styleoptions: authoryear, alphabetic, numeric, ...
%    sorting=none,	% Sorting of the entries
%    natbib=true,	% activate natbib commands
%    hyperref=true,	% activvate hyperref support
%    backref=true,	% activate backrefs
%    isbn=false,		% don't show isbn tags
%    url=false,		% don't show url tags
%    doi=false,		% don't show doi tags
%    urldate=long,	% display type for dates
%    maxbibnames=3,
%    minbibnames=1,
%    maxcitenames=1,
%    mincitenames=1
%    ]{biblatex}
%    \bibliography{References1.bib}
%    \DefineBibliographyStrings{english}{%
%	andothers = {{et\,al\adddot}}, % 'et al.' instead of 'u.a.'
%	backrefpage = {{cited on page}},%
%	backrefpages = {{cited on pages}},%
%    }
% Author names in publication list are consistent 
%\DeclareNameAlias{author}{given-family} }}}

% Some useful commands %%%
% pdf-friendly newline in links
\newcommand{\pdfnewline}{\texorpdfstring{\newline}{ }} 
% Fill the vertical space in a slide (to put text at the bottom)
\newcommand{\framefill}{\vskip0pt plus 1filll}

\renewcommand{\proofname}{\sffamily{Proof}}

% Add custom meeting title and small title
\newcommand{\meetingtitle}[1]{\begin{textblock*}{0.9\paperwidth}
(0.05\paperwidth,0.23\paperheight){\begin{center}#1\end{center}}\end{textblock*}}
\newcommand{\smalltitle}[1]{\begin{textblock*}{0.9\paperwidth}
(0.05\paperwidth,0.65\paperheight){\begin{center}\usebeamerfont{title}\LARGE
\color{ARed}{#1}\end{center}}\end{textblock*}}

% }}}

\title[ALICE Theme]{Simplistic ALICE beamer class theme}
\date[\today]{\small\today}
\author{Maurice Donner}

\begin{document}

\begin{frame}[fragile] % Titlepage {{{
\thispagestyle{empty}
\titlepage
\meetingtitle{Weekly Hardware meeting}
\smalltitle{- Full template -}
\end{frame}

\addtobeamertemplate{navigation symbols}{}{%
    \usebeamerfont{footline}%
    \usebeamercolor[fg]{footline}%
    \hspace{1cm}%
    \textcolor{ABlue}{\insertframenumber/\inserttotalframenumber}
} % }}}

\begin{frame}{Frame Title - Large red letters}{Frame Subtitle - Less large and less red letters}
    This is how normal text looks like.\\
    This is how \textbf{bold} and \textit{italic} text looks like.
    \begin{enumerate}
	\item This is item one of an enumeration list.
	\item And here we have item two of an enumeration list
	    \begin{enumerate}
	    \item Here we have sub-items in a list
	    \item Take another one will ya
	    \end{enumerate}
	\item And a third and final item of our enumeration.
    \end{enumerate}
    \begin{itemize}
	\item test
	\item test 2
	\begin{itemize}
	    \item undertest
	    \item undertest 2
	\end{itemize}
	\item test 3
    \end{itemize}
\end{frame}


\begin{frame}[fragile]{Frame Title without subtitle}
    Text that can be \highlight{highlighted} or \marker{marked} \\[1cm]
    Code should be formatted with \lstinline!\lstinline! \\[1cm]
    This can also be used to highlight \lstinline!KEYWORDS! or names like
    the \lstinline{ALICE} experiment.
\end{frame}

\framecard{white}{ABlue}{Empty card with large text}

\begin{frame}{Text block}
    \tiny
    \lipsum[1-5]
\end{frame}

% Frame with beackground image
% \framepic[0.5]{/home/maurice/Pictures/Wallpapers/FIAB.jpg}{Test}

\end{document}

